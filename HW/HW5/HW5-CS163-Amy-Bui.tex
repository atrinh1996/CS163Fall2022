% -*- LaTeX -*- %%%%%%%%%%%%%%%%%%%%%%%%%%%%%%%%%%%%%%%%%%%%%%%%%%%%%%
%
% Template for scribing COMP163 - Computational Geometry 
%
% Spring, 2004
%
%%%%%%%%%%%%%%%%%%%%%%%%%%%%%%%%%%%%%%%%%%%%%%%%%%%%%%%%%%%%%%%%%%%%%%
%**start of header 

\documentclass [12pt]{article}
\usepackage{epsfig}
\usepackage{enumitem}
\usepackage{amsmath}
\usepackage{amssymb}
\usepackage[color, leftbars]{changebar}

\usepackage{caption}
\usepackage{subcaption}


\usepackage{color}
\definecolor{light-gray}{gray}{0.97} % shade of grey
\definecolor{dkgreen}{rgb}{0,0.6,0}
\definecolor{gray}{rgb}{0.5,0.5,0.5}
\definecolor{mauve}{rgb}{0.58,0,0.82}

%%%%%%%%%%%%%%%%%%%%%%%%%%%%%%%%%%%%%%%%%%%%%%%%%%%%%%%%%%%%%%%%%%%%%%%%
\usepackage{xcolor}
%% https://tex.stackexchange.com/questions/401750/quick-and-short-command-for-coloring-one-word
\newcommand\shorthandon{\catcode`@=\active \catcode`^=\active \catcode`*=\active }
\newcommand\shorthandoff{\catcode`@=12 \catcode`^=7 \catcode`*=12 }
\shorthandon
\def@#1@{\textcolor{red}{#1}}%
\def^#1^{\textcolor{blue}{#1}}%
\def*#1{\string#1}
\shorthandoff
%% useage: \textcolor{red}{text here}
% \shorthandon
% This is a @test@ of the ^emergency^ bro*@dcast system.
% \shorthandoff
%%%%%%%%%%%%%%%%%%%%%%%%%%%%%%%%%%%%%%%%%%%%%%%%%%%%%%%%%%%%%%%%%%%%%%%%


\setlength{\textwidth}{6.5in}
\setlength{\textheight}{9in}
\setlength{\oddsidemargin}{0in}
\setlength{\evensidemargin}{0in}
\setlength{\topmargin}{-0.5in}

\setlength{\parindent}{0pt}

\newtheorem{theorem}{Theorem}[section]
\newtheorem{definition}[theorem]{Definition}
\newtheorem{claim}[theorem]{Claim}
\newtheorem{lemma}[theorem]{Lemma}
\newtheorem{proof}[theorem]{Proof}

\newlength{\toppush}
\setlength{\toppush}{2\headheight}
\addtolength{\toppush}{\headsep}

\usepackage{hyperref}
\hypersetup{
    colorlinks=true,
    linkcolor=blue, % was previously black
    filecolor=magenta,
    urlcolor=blue,
    pdftitle={Template}
}
\urlstyle{same}

%\doheading{2}{title}{Last Revised: January, 2004}
%\htitle{title}

\def\subjnum{Comp 163}
\def\subjname{Computational Geometry}

\def\doheading#1#2#3{\vfill\eject\vspace*{-\toppush}%
  \vbox{\hbox to\textwidth{{\bf} \subjnum: \subjname \hfil Amy Bui}%
    \hbox to\textwidth{{\bf} Tufts University, Fall 2022 \hfil#3\strut}%
    \hrule}}

\newcommand{\htitle}[1]{\vspace*{3.25ex plus 1ex minus .2ex}%
\begin{center}
{\large\bf #1}
\end{center}} 

%%%%%%%%%%%%%%%%%%%%%%%%%%%%%%%%%%%%%%%%%%%%%%%%%%%%%%%%%%%%%%%%%%%

\begin{document}
\doheading{2}{title}{HW 5} 
% \htitle{Homework 1}
% \bigskip 
% \bigskip 
%%%%%%%%%% begin text after this line %%%%%%%%%%%%%%

    %%%%%%%%%%%%%%%%%%%%%%%%%%%%%%%%%%%%%%%%%%%%%%%%%%%%%%%%%%%%%%%%%%%%%%%%%
    \section{Point Location: 2D}
    \label{sec:one}

        \begin{enumerate}[label=\alph*.]
            \item Translate the points of $S \cup \{q\}$ to the origin, where $q$ is at the origin. For each point $p \in S$, draw vector $\overrightarrow{qp}$. For the set of vectors made about the origin at $q$, if 1) there exists two adjacent vectors that can make an obtuse angle with no other vector bisecting it, then $q$ is outside $\mathcal{CH}(S)$; 2) if two adjacent vectors make $180^{o}$ angle that isn't bisect by another vector, then $q$ is colinear to those two points and all three are on $\mathcal{CH}(S)$; and 3) all pairs of adjacent vectors make acute angles, then $q$ is inside $\mathcal{CH}(S)$. This is done in linear time because we check adjacent vectors. 
            
            \item We know $q$ is inside $\mathcal{CH}(S)$. Suppose we presorted $S$ and know the points are ordered by x-coordinates, which itself takes $O(n\log n)$ time. Split the set of $S$ across two half planes divided by the horizontal line going through $q$\footnote{Diane had gone over this method in OH.}. Take the left most and right most point in the upper plane and in the lower plane, and for each of those (at most) four points, make the vector from $q$ to that extreme point. Pick three of those (at most) four points, and at least one of those triples will form a triangle $\Delta xyz$ that contains $q$. Since we have at most $\binom{4}{3}$ triangles to check to check, this takes constant time to check this many point inclusions in a triangle. This works since $q$ is already known to be within the convex hull; since we found (at most) 4 exteme points, these are in $\mathcal{CH}(S)$, and $q$ will be contained in a face made by these vertices.
            
            \item We know $q$ is outside $\mathcal{CH}(S)$, so there exists some region ``between'' $q$ and $S$ that is empty and keeps them to opposite sides. We can find a support line of $\mathcal{CH}(S)$ that also keeps $q$ to the opposite side of the line. Similar to part a, for each point $p \in S$, draw line segment $\overline{qp}$. We can find the shortest line segment in $O(n)$ time. Since this $p$ is the ``closest'' to $q$ out of all other points in $S$, itis  some extreme point and so $p\in \mathcal{CH}(S)$. Draw the line perpendicular to this shortest $\overline{qp}$, and that is the support line that keeps $S$ and $q$ on opposite sides. No other points of $S$ would be on the $q$-side of the line, because then such a point would closer to $q$ than the closest point $p$, a contradition; additionally, since $p$ is on the convex hull, this other closest point $p'$ must also be on the convex hull; however, $p'$ existance on the hull and on the $q$ side of the line is a contradiction because it makes the polygon concave, and therefore not a convex hull. 
         
            \item Suppose we have calculated the $\mathcal{CH}(S)$, the three tasks can be done as:
                \begin{enumerate}[label=\textbf{part \alph*}]
                    \item It takes $O(\log n)$ time to do point inclusion with a convex polygon. 
                    \item It takes $O(n)$ time to triangulate a convex hull. Supposing we store the triangulation in a DCEL, it can take $O(\log n)$ time to find the face in the DCEL that $q$ belongs, and we report the three points of that triangle face, accessible in constant time.
                    \item If $\mathcal{CH}(S)$ is in a dynamic data structure, it takes $O(\log n)$ to find a bridge $\overline{qp}$ from $q$ to $\mathcal{CH}(S)$, where $p \in \mathcal{CH}(S)$. If $\overline{qp} \Rightarrow y = mx + b$, the support line that separates $S$ ad $q$ is ``tilted'' $\overline{qp}$, i.e. $y = (m \pm \epsilon_1)x + (b \pm \epsilon_2)$, that includes point $p$. 
                \end{enumerate}
            
        \end{enumerate}
        
    % \pagebreak
    % END %%%%%%%%%%%%%%%%%%%%%%%%%%%%%%%%%%%%%%%%%%%%%%%%%%%%%%%%%%%%%%%%%%%

    %%%%%%%%%%%%%%%%%%%%%%%%%%%%%%%%%%%%%%%%%%%%%%%%%%%%%%%%%%%%%%%%%%%%%%%%%
    \section{Point Location: 3D}
    \label{sec:two}

        \begin{enumerate}[label=\alph*.]
            \item For each $p \in S$ and $q$, translate the points about the origin, putting $q$ at the origin. Using $q$, create three hyperplanes $H_{q,1}$, $H_{q,2}$, and $H_{q,3}$, that are orthogonal to the $x_1$-, $x_2$-, and $x_3$-axis, respectively. In $O(n)$ time, check if all points are to one side of a hyperplane. If at least one hyperplane keeps points of $S$ to one side, then $q$ is outside the $\mathcal{CH}(S)$; otherwise, $q$ is inside the  $\mathcal{CH}(S)$.  
            
            \item We know that $q$ is inside the $\mathcal{CH}_{\text{3D}}(S)$. Sort the points of $S$ in $O(n\log n)$ time, then find the at most 6 extreme points, leftmost and rightmost of each axis. Since these are extreme points, they are on $\mathcal{CH}_{\text{3D}}(S)$. We check at most the $\binom{6}{4}$ sets of four points that make a tetrahedral inside the hull to see if it contains $q$ in $O(\log n)$ time. Report the set of four points that does contain $q$. 
            
            \item We know that $q$ is outside the $\mathcal{CH}_{\text{3D}}(S)$. We take $O(n)$ time to find the 3 or 4 shortest $\overline{qp}$, for some $p\in S$. These points together form a face on the convex hull of S that is closest to $q$. A hyperplane made from these points form the support plane that separates $q$ from $S$. 
            
            \item Suppose we have calculated the $\mathcal{CH}(S)$, the three tasks can be done:
            %  in linear time using linear programming as a subroutine\footnote{Diane had gone over this in OH.}:
            
            \begin{enumerate}[label=\textbf{part \alph*}]
                \item Lay the 3D convex hull on the xy-plane, yz-plane, and xz-plane. The result is a triangulation in 2D plane of the 2D hull. We can do the three point inclusion for $q$ in these planes in order to determine if $q$ is in the hull in $O(\log n)$ time. If it is not in the hull in any of the 2D planes, it is not in the 3D hull. 
                
                \item Using linear programming we can find the tetrahedron containing $q$ in linear time\footnote{Diane had gone over this in OH.}. Laying the hull on a 2D plane gives the 2D hull and a triangulation. Select the three points of a triangle $q$ is located in and the one point closets to $q$ not on that face, and these four points are the four points of the tetrahedron that contain $q$ in 3D. 
                
                \item Using linear programming we can find the hyperplane that separates $q$ and $S$ in linear time\footnotemark[2], by finding the facet of the convex hull of $S$ that is closest to $q$. That facet turned into a plane is the support plane that separates $q$ and $S$.
            \end{enumerate}

        \end{enumerate}
        
    % \pagebreak
    % END %%%%%%%%%%%%%%%%%%%%%%%%%%%%%%%%%%%%%%%%%%%%%%%%%%%%%%%%%%%%%%%%%%%



\begin{thebibliography}{1}
    \bibitem[1]{officehours}Jake and Diane's office hours.
    \bibitem[2]{berg08}Mark de Berg, Otfried Cheong, Marc van Kreveld, and Mark Overmars. 2008. Computational Geometry: Algorithms and Applications (3rd ed. ed.). Springer-Verlag TELOS, Santa Clara, CA, USA.
\end{thebibliography}
%%%%%%%%%%%%%%%%%%%%%%%%%%%%%%%%%%%%%%%%%%%%%%%%%%%%%%%%%%%%%%%%%%%%%%
\end{document}
%%%%%%%%%%%%%%%%%%%%%%%%%%%%%%%%%%%%%%%%%%%%%%%%%%%%%%%%%%%%%%%%%%%%%%

